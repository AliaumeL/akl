% This is a demo document for the akltex package
\documentclass{article}
\usepackage{hyperref}
\usepackage{url}

\usepackage{namespc}


% Loading the package with the "rewrite" mode
% updates all the links to be handled by `akl`
\usepackage[rewrite]{akltex}

\author{Aliaume Lopez}
\title{Demo akl document}

\begin{document}

\maketitle

%% TODO: this is problematic
%% because it does not allows
%% to have multiple names for
%% the same entity.
%
% -> could we have an `entity` latex package?
%
% \NewEntity{name,same as={other entity}}
%
% \GetEntity{name} -> returns a unique identifier
% \SameEntities{name,name} -> boolean
% 
\akldef{
    key=douéneautabot2022pebble,
    name=Super def,
    url={arxiv:2104.14019?page=11&dest=theorem.5.37}
}


%%% THE FIRST PART OF THE DOCUMENT
%%% ILLUSTRATES THE SYNTAX FOR DEFINING
%%% CUSTOM ENTRYPOINTS OF DOCUMENTS

% Adding a figure as part of a paper
\aklset{romano2023modeling}{Figure 1}{https://arxiv.org/abs/2305.15355?page=5&dest=figure.1}


% Referencing a section, using the key-value syntax
\akldef{
    key=romano2023modeling,
    name=Section 4,
    url={https://arxiv.org/abs/2305.15355?page=9&dest=section.4}
}

% Leveraging the `withakl` command
% and specifying a custom url allows
% to "batch insert" relevant parts of a document.
%
% Here we demonstrate the usage with specific pages.
\withakl[https://arxiv.org/abs/2305.15355]{romano2023modeling}{
    % Figure 8 of the document
    \aklsetd{name=Figure 8,page=11,dest={figure.8}}
    % Reference to some "unnamed" destinations
    % by only specifying the page number
    \aklsetd{name={Third bullet point page 10}, page=10}
    \aklsetd{name={p. 2}, page=2}
    \aklsetd{name={p. 1}, page=1}
    \aklsetd{name={p. 7}, page=7}
}

%%% THIS SECOND PART ILLUSTRATES HOW
%%% THE CUSTOM DESTINATIONS INTRODUCED
%%% IN THE PREVIOUS PART CAN BE USED.


It is possible to directly get a value, for instance using
\aklget{romano2023modeling}{Figure 1}. This may be out of context, and you may
prefer to cite the document along, for instance using \aklcite[Figure
1]{romano2023modeling}. If you want to cite a page, you can too \aklcite[p.
2]{romano2023modeling}. Let us just check that it works for section 4:
\aklcite[Section 4]{romano2023modeling}.


Now, we can go a bit further and cite several parts of the document as follows
\cite[\withakl{romano2023modeling}{ see  \akl{p. 7} and \akl{Figure 8}
}]{romano2023modeling}. To not repeat yourself, you can use \acite[\akl{Third
bullet point page 10} and \akl{Section 4}]{romano2023modeling}.

\clearpage

\aklcite[Super def]{douéneautabot2022pebble}



\bibliographystyle{plain}
\bibliography{bibliography}

\end{document}
